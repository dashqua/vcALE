Material Motion $n^\circ 1$:
\begin{equation}
	\vect{\psi} = \psi_\beta(\cchi,t) = 
	\begin{pmatrix}
		\chi_1 + \beta \sin(2\pi\chi_1) \sin(\pi\chi_2/3)\sin(\pi t/T) \\
		
		\chi_2 + 5\beta \sin(2\pi\chi_1) \sin(\pi\chi_2/3)\sin(2\pi t/T) \\
		
		\chi_3
	\end{pmatrix},
\end{equation}
with $T=2$ and $\beta$ an amplitude parameter defined for each simulation.

Material Motion $n^\circ 2$:
\begin{equation}
	\vect{\psi} = \psi_\beta(\cchi,t) = 
	\begin{pmatrix}
		\chi_1 + \beta \sin^2(2\pi\chi_1) \sin^2(\pi\chi_2/3)\sin(\pi t/T) \\
		
		\chi_2 + 5\beta \sin^2(2\pi\chi_1) \sin^2(\pi\chi_2/3)\sin(2\pi t/T) \\
		
		\chi_3
	\end{pmatrix},
\end{equation}
\begin{equation}
      \vect{w} = 
      \begin{pmatrix}
          \frac{\beta\pi}{T}  \sin^2(2\pi \chi_1)  \sin^2(2\pi \chi_1)   \cos(\pi t/T) \\
           \frac{10.0\beta\pi}{T}  \sin^2(2\pi \chi_1)  \sin^2(\pi Y/3.0)  \cos(2\pi t/T) \\
           0.0
\end{pmatrix},
\end{equation}
\begin{equation}
\mat{F}_{\psi} = 
\begin{pmatrix}
1 + 4\beta\pi  \sin(2\pi X)  \cos(2\pi X)  \sin(\pi Y/3.0)   \sin(\pi Y/3.0)   \sin(\pi t/T ) &
2 \beta  \pi/3.0   \sin(2 \pi X)   \sin(2 \pi X)   \cos(\pi Y/3.0)   \sin(\pi Y/3.0)   \sin(\pi t/T ) &
0 \\
20.0 \beta  \pi  \cos(2 \pi X)   \sin(2 \pi X)   \sin(\pi Y/3.0)   \sin(\pi Y/3.0)   \sin(2 \pi t/T ) &
1 + 10.0 \beta  \pi/3.0   \sin(2 \pi X)   \sin(2 \pi X)   \cos(\pi Y/3.0)   \sin(\pi Y/3.0)   \sin(2 \pi t/T ) &
0,\\
0 &
0 &
1
);
\end{pmatrix}
\end{equation}

\newpage
\section{Conservation Laws}
%\subsection{Relations}
\subsection{Neo Hookean model}
\begin{equation}
	p = \kappa \left( det(\mat{trueF}) - 1 \right), \quad \textrm{with } \mat{trueF} = \mat{F} \mat{F}_{\psi}^{-1}
\end{equation}
\subsection{Set of conservation laws}
\begin{subequations}
\begin{align}
	\frac{\partial \mat{F}}{\partial t} &= \DIV_{\cchi}\left( \hat{\vect{v}} \otimes \mat{I} \right) \\ 
	\frac{\partial \J}{\partial t} &= \mat{H}_{\psi} \bm{:} \GRAD_{\cchi}\left( \vect{w} \right) \\
	\frac{\partial \mat{F}_{\psi}}{\partial t} &= \DIV_{\cchi}\left( \vect{w} \otimes \mat{I} \right) \\
	\frac{\partial \tilde{\vect{p}}}{\partial t} &= \DIV_{\cchi}\left( \mat{P}\mat{H}_{\psi}\right) {~}+{~} \GRAD_{\cchi} \left(  \vect{p}_R \right) \left( \mat{H}_{\psi}^{T}\vect{w} \right)  {~}+{~} \vect{p}_R \left[\mat{H}_{\psi} \bm{:} \GRAD_{\cchi}\left( \vect{w} \right) \right]
\end{align}
\end{subequations}

%\section{Time discretization}
%\subsection{Runge-Kutta 2 stage time discretization}

\section{Finite Volume Method spatial discretization}
%\subsection{Discretised Gradient}
%\begin{equation}
%	\GRAD_{\cchi} \vect{B}_{\cchi}\rvert_{a} = \frac{1}{\Omega_{\cchi}^a} \left[ \sum\limits_{b\in\Lambda_a} \vect{B}^{Ave} \otimes \vect{C}_{\cchi}^{ab} {~}+{~} \sum\limits_{\gamma\in\Lambda_a^B} \vect{B}_{a}^{\gamma} \otimes \vect{C}_{\cchi}^{\gamma} \right] 
%\end{equation}
%
\subsection{Discretised set of conservation laws}
\begin{subequations}
\begin{align}
	\Omega_{\cchi}^a \frac{\partial \mat{F}^a}{\partial t} &= \sum\limits_{b\in\Lambda_a} \hat{\vect{v}}^{Ave} \otimes \vect{C}_{\cchi}^{ab} {~}+{~} \sum\limits_{\gamma\in\Lambda_a^B} \hat{\vect{v}}_{a}^\gamma \otimes \vect{C}_{\cchi}^{\gamma} \\
	%
	\Omega_{\cchi}^a \frac{\partial \J^a}{\partial t} &= \mat{H}_{\psi} \bm{:} \left( \sum\limits_{b\in\Lambda_a} \vect{w}^{Ave} \otimes \vect{C}_{\cchi}^{ab} \right) \\
	%
	\Omega_{\cchi}^a \frac{\partial \mat{F}_{\psi}^a}{\partial t} &= \sum\limits_{b\in\Lambda_a} \vect{w}^{Ave} \otimes \vect{C}_{\cchi}^{ab} {~}+{~} \sum\limits_{\gamma\in\Lambda_a^B} \vect{w}_{a}^{\gamma} \otimes \vect{C}_{\cchi}^{\gamma} \\
	%
	\Omega_{\cchi}^a \frac{\partial \tilde{\vect{p}}^a}{\partial t} &= \sum\limits_{b\in\Lambda_a} \left( \mat{PH_{\psi}} \right)^{Ave} \vect{C}_{\cchi}^{ab} {~}+{~} \sum\limits_{\gamma\in\Lambda_a^B} \vect{t}_{\cchi,a}^\gamma  \norm{\vect{C}_{\cchi}^{\gamma}}    +\sum\limits_{b\in\Lambda_a} \vect{D}_{\cchi,ab} \nonumber\\
	%
	&+{~} \left( \sum\limits_{b\in\Lambda_a} \vect{p}_R^{Ave} \otimes \vect{C}_{\cchi}^{ab} \right) \mat{H}_{\psi}^{T} \vect{w} {~}+{~} \left( \sum\limits_{\gamma\in\Lambda_a} \vect{p}_{R,\gamma} \otimes \vect{C}_{\cchi}^{\gamma} \right) \mat{H}_{\psi}^{T} \vect{w} \nonumber\\
	%
	&+{~}  \vect{p}_R \left[ \mat{H}_{\psi} \bm{:} \left( \sum\limits_{b\in\Lambda_a} \vect{w}^{Ave} \otimes \vect{C}_{\cchi}^{ab} \right) \right] {~}+{~} \vect{p}_R \left[ \mat{H}_{\psi} \bm{:} \left( \sum\limits_{\gamma\in\Lambda_a^B} \vect{w}_a^{\gamma} \otimes \vect{C}_{\cchi}^{\gamma} \right) \right]
\end{align}
\end{subequations}

\newpage

\begin{subequations}
\begin{equation}
	\mat{P}_{mat} = \mat{P}_{mat} \left( \mat{F}_{mat} \right)
\end{equation}
\begin{equation}
	\vect{F}_{b,\vect{w}} = \rho_R \dot{\vect{w}} - \DIV \left( \mat{P}_{mat}\mat{H}_{\psi} \right)
\end{equation}
\begin{equation}
	\vect{w}^{n+1} = \vect{w}^n + \Delta t \left[ \DIV \left( \mat{P}_{mat}\mat{H}_{\psi} \right) + \vect{F}_{b,\vect{w}} \right]; 
\end{equation}

\end{subequations}


\newpage
\subsection{Algorithms}
\begin{algorithm}[H]
	%\KwData{$c^{p}_{R}, c^{s}_{R}$.} 
	\KwResult{\textbf{rhsLm}, \textbf{rhsF}. }
	$\vect{p}_R^n = \J_{\psi}^{-1} \vect{p}_{\cchi}^n$ \\
	\bigbreak 
	$\vect{rhsLm}_{a}^{*} = \left( \sum\limits_{b\in\Lambda_a} \vect{tC}_b^{n} ||\vect{C}_{\cchi}^{ab}|| \right) + \left( \sum\limits_{b\in\Lambda_a} \vect{tC2}_b^{n} ||\vect{C}_{\cchi}^{ab}|| \right) \mat{H}_{\psi}^T \vect{w} + \vect{p}^n_R \left(\mat{H}_{\psi}\bm{:} \left[ \sum\limits_{b\in\Lambda_a} \vect{tC3}_b^{n} ||\vect{C}_{\cchi}^{ab}|| \right]\right) $ \\
	\bigbreak
	Boundary value of the first RHS term.
	$\vect{tC}_{loc}^\gamma = LocalAverage(\hat{\mat{P}} \vect{N}_{\cchi}^\gamma)$ or apply the prescribed traction boundary condition. The nodal value is updated as $\vect{rhsLm}_\gamma^{*} = \vect{rhsLm}_\gamma^{n} + \vect{tC}_{loc}^\gamma ||\vect{C}_{\cchi}^\gamma||$. \\
	\bigbreak
	Boundary value of the second RHS term.
	$\vect{tC2}_{loc}^\gamma = LocalAverage(\vect{p}_R \vect{N}_{\cchi}^\gamma)$. The nodal value is updated as $\vect{rhsLm}_\gamma^{*} = \vect{rhsLm}_\gamma^{n} + \left(\vect{tC2}_{loc}^\gamma ||\vect{C}_{\cchi}^\gamma||\right) \mat{H}_{\psi,\gamma}^T \vect{w}_\gamma$. \\
	\bigbreak
	Boundary value of the third RHS term.
	$\vect{tC3}_{loc}^\gamma = LocalAverage(\vect{w} \vect{N}_{\cchi}^\gamma)$. The nodal value is updated as $\vect{rhsLm}_\gamma^{*} = \vect{rhsLm}_\gamma^{n} + \vect{p}_{R, \gamma}\left( \mat{H}_{\psi,\gamma} \bm{:} (\vect{tC3}_{loc}^\gamma ||\vect{C}_{\cchi}^\gamma||)\right)$. \\
	\bigbreak
	Volume integration: $\vect{rhsLm}_a^{*} = \vect{rhsLm}_a^{*} / \Omega_a$. \\
	\bigbreak
	Update of displacements $\vect{x}^{*} = \vect{x}^{n} + \Delta t \hat{\vect{v}}^n$ and linear momentum $\vect{p}_{\cchi}^{*} = \vect{p}_{\cchi}^{n} + \J_{\psi}\J^{-1} \Delta t ~\vect{rhsLm}^{*}$.
	\bigbreak
	Application of boundary conditions for the linear momentum. \\
	\bigbreak
	Update of variables: 
	\begin{align*}
		\mat{trueF}^{*} &= \mat{F}^{*} \mat{F}_{\psi}^{-1,*} \\
		\mat{P}^{*} &= \mat{P}^{*}(\mat{trueF^{*}}) \\
		\vect{p}_{R}^{*} &= \J_{\psi}^{-1} \vect{p}^{*} \\
		\vect{v}^{*} &= \vect{p}_{R}^{*} / \rho \\
		\hat{\vect{v}}^{*} &= \vect{v}^{*} + \mat{trueF}^{*} \vect{w}^{n} \\
		\hat{\mat{P}}^{*} &= \mat{P}^{*} \mat{H}_{\psi}
	\end{align*}\\
	\bigbreak
	Computation of coefficient $\Lambda_H^n$ and update of $U_p^{n}$ and $U_s^{n}$. \\
	\bigbreak
	Update of the averaged values for computing the right hand sides 
	\begin{align*}
		\vect{tC}^{*} &= \hat{\mat{P}}^{Ave,*} \vect{N} + 0.5 S^{*} [\vect{v}^{*}] \\
		\vect{tC2}^{*} &= \vect{p}_{R}^{Ave,*} \otimes \vect{N} \\
		\vect{tC3}^{*} &= \vect{w}^{,Ave,n} \otimes \vect{N}
	\end{align*}
	
	
%	Update of $\vect{w}, \mat{F}_{\psi}, \J_{\psi}, \mat{H}_{\psi}$ \\	
%	$\mat{F} = \mat{F}_{\phi}\mat{F}_{\psi}^{-1}$ \\
%	$\mat{P} = \mat{P}(\mat{F})$ \\
%	$\vect{v} = \J_{\psi}^{-1} \vect{p}_{\x}/\rho$ \\
%	$\hat{\vect{v}} = \vect{v} + \left[ \left( \mat{F}_{\phi}\mat{F}_{\psi}^{-1}\right) \vect{w} \right]$ \\
%	$\hat{\mat{P}} = \left[ \mat{P} + \left( (\J_{\psi}^{-1} \vect{p}_{\x})\otimes \vect{w}   \right) \right]\mat{H}_{\psi} $ \\
%	$\vect{vC}$, point to edge interpolation of $\hat{\vect{v}}$ \\
%	$\Lambda_{\mat{H}_\phi}^2 = \left( \mat{H}_{\psi}^{Ave} \N_{\cchi} \right) \cdot \left( \mat{H}_{\psi}^{Ave} \N_{\cchi} \right)$ \\
%	$c^{p}_{\x} = \J_{\psi}^{-1} \left[ \Lambda_{\mat{H}_\phi}c^{p}_{R} - \vect{w}\cdot\left( \mat{H}_{\psi}\N_{\cchi} \right) \right]$ \\
%	$c^{s}_{\x} = \J_{\psi}^{-1} \left[ \Lambda_{\mat{H}_\phi}c^{s}_{R} - \vect{w}\cdot\left( \mat{H}_{\psi}\N_{\cchi} \right) \right]$  \\
%	$\mat{tC} = \hat{\mat{P}}^{Ave} \N_{\x} + \frac{1}{2} Smat(c^{p}_{\x},c^{s}_{\x}) \left( \vect{p}_{\x}^{+}-\vect{p}_{\x}^{-}\right)$ \\
%	$\vect{rhsLm} =  \sum\limits_{b\in\Lambda_a} \left( \vect{tC} \; ||C^{\x}_{ab}|| \right)$ \\
%	$\mat{rhsF} =  \sum\limits_{b\in\Lambda_a} \left( \vect{vC} \otimes \vect{C}^{\x}_{ab} \right)$ \\
	\caption{Computation of the right hand side of the linear momentum conservation law for the first Runge-Kutta stage.}
\end{algorithm}


%
%
%\newpage
%\newpage
%\bigbreak
%\bigbreak
%\bigbreak
%\bigbreak
%\newpage
%
%\begin{algorithm}[H]
%	\KwData{$c^{p}_{R}, c^{s}_{R}$.} % referential wave speeds
%	\KwResult{\textbf{rhsLm}, \textbf{rhsF}. }
%	%initialization\;
%	Update of $\vect{w}, \mat{F}_{\psi}, \J_{\psi}, \mat{H}_{\psi}$ \\
%	$\mat{F} = \mat{F}_{\phi}\mat{F}_{\psi}^{-1}$ \\
%	$\mat{P} = \mat{P}(\mat{F})$ \\
%	$\vect{v} = \J_{\psi}^{-1} \vect{p}_{\x}/\rho$ \\
%	$\hat{\vect{v}} = \vect{v} + \left[ \left( \mat{F}_{\phi}\mat{F}_{\psi}^{-1}\right) \vect{w} \right]$ \\
%	$\hat{\mat{P}} = \left[ \mat{P} + \left( (\J_{\psi}^{-1} \vect{p}_{\x})\otimes \vect{w}   \right) \right]\mat{H}_{\psi} $ \\
%	$\vect{vC}$, point to edge interpolation of $\hat{\vect{v}}$ \\
%	$\Lambda_{\mat{H}_\phi}^2 = \left( \mat{H}_{\psi}^{Ave} \N_{\cchi} \right) \cdot \left( \mat{H}_{\psi}^{Ave} \N_{\cchi} \right)$ \\
%	$c^{p}_{\x} = \J_{\psi}^{-1} \left[ \Lambda_{\mat{H}_\phi}c^{p}_{R} - \vect{w}\cdot\left( \mat{H}_{\psi}\N_{\cchi} \right) \right]$ \\
%	$c^{s}_{\x} = \J_{\psi}^{-1} \left[ \Lambda_{\mat{H}_\phi}c^{s}_{R} - \vect{w}\cdot\left( \mat{H}_{\psi}\N_{\cchi} \right) \right]$  \\
%	$\mat{tC} = \hat{\mat{P}}^{Ave} \N_{\x} + \frac{1}{2} Smat(c^{p}_{\x},c^{s}_{\x}) \left( \vect{p}_{\x}^{+}-\vect{p}_{\x}^{-}\right)$ \\
%	$\vect{rhsLm} =  \sum\limits_{b\in\Lambda_a} \left( \vect{tC} \; ||C^{\x}_{ab}|| \right)$ \\
%	$\mat{rhsF} =  \sum\limits_{b\in\Lambda_a} \left( \vect{vC} \otimes \vect{C}^{\x}_{ab} \right)$ \\
%	\caption{Computation of Right Hand Sides.}
%\end{algorithm}
%
%\begin{algorithm}[H]
%	\KwData{\textbf{rhsLm}, \textbf{rhsF}. } % referential wave speeds
%	\KwResult{$\vect{p}_{\x}^n, \mat{F}_{\phi}^n, \vect{u}^{n}, \vect{u}_{\vect{w}}^{n}$}
%	%initialization\;
%	$\vect{v} = \J_{\psi}^{-1} \vect{p}_{\x}/\rho$ \\
%	$\hat{\vect{v}} = \vect{v} + \left[ \left( \mat{F}_{\phi}\mat{F}_{\psi}^{-1}\right)\cdot\vect{w} \right]$ \\
%	$\x \mathrel{{+}{=}} \Delta t \; \hat{\vect{v}}$ \\
%	$\x_{\vect{w}} \mathrel{{+}{=}} \Delta t \; \vect{w}$ \\
%	$\vect{p}_{\x} \mathrel{{+}{=}} \Delta t \; \vect{rhsLm}$ \\
%	$\mat{F} \mathrel{{+}{=}} \Delta t \; \mat{rhsF}$ \\
%	$\vect{p}_{\x}^n = \frac{1}{2}\left( \vect{p}_{\x}^{n-1} + \vect{p}_{\x}^n \right)$ \\
%	$\mat{F}_{\phi}^n = \frac{1}{2}\left( \mat{F}_{\phi}^{n-1} + \mat{F}_{\phi}^n \right)$ \\
%	$\x^n = \frac{1}{2}\left( \x^{n-1} + \x^n \right)$ \\
%	$\x_{\vect{w}}^n = \frac{1}{2}\left( \x_{\vect{w}}^{n-1} + \x_{\vect{w}}^n \right)$ \\
%	$\vect{u}^{n} = \x^n - \X$ \\
%	$\vect{u}_{\vect{w}}^{n} = \x_{\vect{w}}^n - \X$
%	\caption{Use of Right Hand Sides.}
%\end{algorithm}
%
%\bigbreak


%\newpage
%\subsection{Energy equation}
%
%\begin{subequations}
%Theoretical Energy:
%\begin{equation}
%	E_{\cchi}^{theo} = \left[ \frac{1}{2} \frac{\vect{p}_{\x}\cdot \vect{p}_{\x}}{\rho_R} \right] + \left[ \frac{\mu}{2} \J^{-2/3} (\mat{F}\bm{:}\mat{F}) - \frac{3\mu}{2} + \frac{\kappa}{2}(\J-1)^{2}\right]
%\end{equation}
%Solved Energy
%\begin{equation}
%	\frac{\partial E_{\cchi}}{\partial t} = \DIV_{\cchi} \left( (\mat{P}\mat{H}_{\psi})^T\vect{v} \right) + \DIV_{\cchi} \left( \frac{E_{\cchi}}{\J_{\psi}}\mat{H}_{\psi}^T \vect{w} \right)
%\end{equation}
%\begin{equation}
%	\frac{\partial E_{\cchi}}{\partial \vect{n}} = 0; \qquad E_{\cchi,0} = 17\times10^6\;  [kg.m^{-1}.s^{-2}]
%\end{equation}
%\end{subequations}

\newpage
\subsection{Conservation of the spatial Jacobian}
Initially:
\begin{equation}
	\frac{\partial \J_{\phi}}{\partial t} = \DIV_{\cchi} \left( \mat{H}_{\phi}^T \hat{\vect{v}} \right)
\end{equation}
Using the velocity relation $\hat{\vect{v}} = \vect{v} + \mat{F}\vect{w}$ and the involution $\DIV_{\cchi}\mat{H}_{\phi} = \vect{0}$, the above equation yields
\begin{equation}
	\frac{\partial \J_{\phi}}{\partial t} = \mat{H}_{\phi} \bm{:} \GRAD_{\cchi}\vect{v} + \DIV_{\cchi} \left( \mat{H}_{\phi}^T \mat{F} \vect{w} \right)
\end{equation}
\begin{subequations}
The VC-FVM discretization is as follows:
\begin{equation}
	V_a^{\cchi} \frac{\partial \J_{\phi}^a}{\partial t} = \mat{H}^a_{\phi} \bm{:} \GRAD_{\cchi}\vect{v}_a + \DIV_{\cchi} \left( \mat{H}^T_{\phi} \mat{F} \vect{w} \right)_a + stab_a
\end{equation}
with:
\begin{equation}
    \mat{H}^a_{\phi} \bm{:} \GRAD_{\cchi}\vect{v}_a =  \mat{H}^a_{\psi} \bm{:} \left( \sum\limits_{b\in\Lambda_a} \vect{v}_a^{Ave} \otimes \vect{C}_{\cchi}^{ab} \right) + \mat{H}^a_{\phi} \bm{:} \left( \sum\limits_{\gamma\in\Lambda_a^B} \vect{v}_a^{\gamma} \otimes \vect{C}_{\cchi}^{\gamma} \right) 
\end{equation}	
\begin{equation}
	\DIV_{\cchi} \left( \mat{H}^T_{\phi} \mat{F} \vect{w} \right)_a = \sum\limits_{b\in\Lambda_a} \left( \mat{H}^T_{\phi} \mat{F} \vect{w} \right)^{Ave} \vect{C}_{\cchi}^{ab} + \sum\limits_{\gamma\in\Lambda_a^B} \left( \mat{H}^T_{\phi} \mat{F} \vect{w} \right)^{\gamma} \vect{C}_{\cchi}^{\gamma} 
\end{equation}
\begin{equation}
	stab_a = \sum\limits_{b\in\Lambda_a} \left( \frac{1}{2} \mat{S}_{ab}^{J_{\phi}} (p^+ - p^-) \mat{H}_{\phi}^{Ave} \vect{N}_{ab}^{\cchi} \right)\cdot \left( \mat{H}_{\phi}^{Ave} \vect{C}_{ab}^{\cchi} \right)
\end{equation}
\end{subequations}
%
\begin{note}{Patch test} \newline
Let $\vect{v}_{cst}$ be a constant velocity field. \newline
Say $\vect{v} = \vect{v}_{cst}$, then $\mat{F} = \mat{H} = \mat{I}$. \newline
As a result, $\mat{F}_{\psi} = \mat{F}_{\phi}$ and $\mat{H}_{\psi} = \mat{H}_{\phi}$. \newline 
Recalling the spatial GCL, and using $\hat{\vect{v}} = \vect{v} + \mat{F}\vect{w}$, it yields
\begin{equation*}
    \frac{\partial \J_{\phi}}{\partial t} = \DIV_{\cchi} \left( \mat{H}_{\phi}^T  \hat{\vect{v}}\right) = \DIV_{\cchi} \left( \mat{H}_{\phi}^T  \vect{w}\right)
\end{equation*}
Recalling the material GCL, and using $\hat{\vect{v}} = \vect{v} + \mat{F}\vect{w}$, it yields
\begin{equation*}
	\frac{\partial \J_{\psi}}{\partial t} = \DIV_{\cchi} \left( \mat{H}_{\psi}^T  \vect{w}\right)
\end{equation*}
And it is obtained that a patch satisfying the constant velocities condition also fulfils
\begin{equation*}
	 \frac{\partial \J_{\phi}}{\partial t} = \DIV_{\cchi} \left( \mat{H}_{\psi}^T  \vect{w}\right) = \frac{\partial \J_{\psi}}{\partial t}
\end{equation*}
\end{note}
%
\subsection{Conservation of the material Jacobian}
The conservation of the material Jacobian is given subsequently given as:
\begin{equation}
	\frac{\partial \J_{\psi}}{\partial t} = \DIV \left( \mat{H}_{\psi}^T \vect{w} \right)
\end{equation}
%
\subsection{Conservation of the spatial Deformation Gradient}
In low order discretization, the conservation of the Deformation Gradient aims at avoiding locking (especially in a nearly-incompressible scenario). It yields 
\begin{equation}
	\frac{\partial \mat{F}_{\phi}}{\partial t} = \GRAD_{\cchi} \left( \hat{\vect{v}} \right)
\end{equation}
and can alternatively written as 
\begin{equation*}
	\frac{\partial \mat{F}_{\phi}}{\partial t} = \DIV_{\cchi} \left( \hat{\vect{v}} \otimes \mat{I} \right)
\end{equation*}
\begin{subequations}
The VC-FVM discretisation is as follows:
\begin{equation}
V_a^{\cchi} \frac{\partial \mat{F}_{\phi}^a}{\partial t} = 
\GRAD_{\cchi} \left( \hat{\vect{v}} \right)_a
\end{equation}
where
\begin{equation}
	\GRAD_{\cchi} \left( \hat{\vect{v}} \right)_a = \sum\limits_{b\in\Lambda_a} \hat{\vect{v}}_a \otimes \vect{C}_{\cchi}^{ab}  +  \sum\limits_{\gamma\in\Lambda_a} \hat{\vect{v}}_{\gamma} \otimes \vect{C}_{\cchi}^{\gamma} 
\end{equation}
\end{subequations}
%
\subsection{Conservation of the material Deformation Gradient}
The material Deformation Gradient, which corresponds to the Deformation Gradient of the material motion, is also a conserved quantity considered here. Its associated conservation law is defined as
\begin{equation}
	\frac{\partial \mat{F}_{\psi}}{\partial t} = \GRAD_{\cchi} \left( \vect{w} \right)
\end{equation}
and can alternatively written as 
\begin{equation*}
	\frac{\partial \mat{F}_{\psi}}{\partial t} = \DIV_{\cchi} \left( \vect{w} \otimes \mat{I} \right)
\end{equation*}
\begin{subequations}
The VC-FVM discretisation is as follows:
\begin{equation}
	V_a^{\cchi}\frac{\partial \mat{F}_{\psi}^a}{\partial t} = \GRAD_{\cchi} \left( \vect{w} \right)_a
\end{equation}
where
\begin{equation}
	\GRAD_{\cchi} \left( \vect{w} \right)_a = \sum\limits_{b\in\Lambda_a} \vect{w}^{Ave}_a \otimes \vect{C}_{\cchi}^{ab} + \sum\limits_{\gamma\in\Lambda_a^B} \vect{w}_{\gamma} \otimes \vect{C}_{\cchi}^{\gamma}
\end{equation}
\end{subequations}
%
\subsection{Conservation of the material velocity}
The evolution of the material velocity $\vect{w}$ is also governed by a conservation law.
A first approach is to circumvent this law and to provide the code with a manufactured right hand side, corresponding to an aimed analytical solution. It then yields
\begin{equation}
  \frac{\partial \vect{w}}{\partial t} = \dot{\vect{w}}
\end{equation}
where $\dot{\vect{w}}$ is a given acceleration term.

A second approach is to solve the following conservation law:
\begin{equation}
	\frac{\partial \vect{w}}{\partial t} = \DIV_{\cchi} \left( \mat{P}^\star \right)
\end{equation}
%
$\mat{P}^\star$ is computed according to an averaged potential:
\begin{equation}
	\mat{P}^\star = \frac{1}{2}\mu_\psi\mat{P}_\psi +\frac{1}{2}\mu_\phi\bar{\mat{F}}^T\mat{P}_\phi
\end{equation}

\subsection{Conservation of the linear momentum}
The linear momentum follows the following conservation law:
\begin{equation}
	f
\end{equation}

\subsection{ALE mixed formulation}
\begin{equation}
	\frac{\partial}{\partial t}
	\begin{bmatrix}
		\vect{p}_{\cchi} \\
		\rho\vect{w} \\
		\mat{F}_{\psi} \\
		\mat{F}_{\phi} \\
		\J_{\psi} \\
		\J_{\phi} \\
		\hat{\mat{D}} \\
		\hat{\alpha} \\
		\E_{\cchi}
	\end{bmatrix}
	-
	\DIV_{\cchi}
	\begin{bmatrix}
		\left(\mat{P}\mat{H}_{\phi}\right) + (\frac{\vect{p}_{\cchi}}{\J_{\psi}}\otimes\vect{w})\mat{H}_{\psi} \\
		\mat{P}^\star \\
		\vect{w}\otimes\mat{I} \\
		\hat{\vect{v}}\otimes\mat{I} \\
		\mat{H}_{\psi}^T\vect{w} \\
		\mat{H}_{\psi}^T\mat{F}\vect{w} \\
		\mat{D} \otimes\left(\mat{H}_{\psi}^T\vect{w}\right) \\
		\alpha\mat{H}_{\psi}^T\vect{w} \\
		\left( \mat{P}\mat{H}_{\psi} \right)^T \frac{\vect{p}_{\cchi}}{\rho} + \mat{H}_{\psi}^T \vect{w}\E_{\cchi}
	\end{bmatrix}
    =
    \begin{bmatrix}
    	\vect{0} \\
    	\vect{0} \\
    	\mat{0} \\
    	\mat{0} \\
    	0 \\
    	\mat{H}_{\phi} \bm{:} \GRAD_{\cchi}\vect{v} \\
    	-2\dot{\gamma} \left( \mat{F}^{-1}\frac{\partial \phi}{\partial \tau}\mat{b}^e\mat{F}^{-T} \right) \\
    	\dot{\gamma} \\
    	0
    \end{bmatrix}
	~
	\begin{matrix}
		\leftarrow \text{stab}\\
		\leftarrow \text{stab}\\
		\\
		\\
		\\
		\leftarrow \text{stab}\\
		\\
		\\
		
	\end{matrix}
,
\end{equation}
with $\hat{\mat{D}} = \J_{\psi}\mat{D} = \J_{\psi} \mat{C}^{p,-1}$ and $\mat{C}^{p,-1}$ the inverse of the plastic right Cauchy tensor, $\hat{\alpha} = \J_{\psi}\alpha$ and $\alpha$ the plastic strain, $\vect{p}_R = \frac{\vect{p}_{\cchi}}{\J_{\psi}}$ the material linear momentum, $\vect{v} = \frac{\vect{p}_R}{\rho}$ the material velocity, $\hat{\vect{v}} = \vect{v} + \mat{F}\vect{w}$ the projected spatial velocity, $\hat{\mat{P}} = \mat{P}\mat{H}_{\psi}$ the projected first Piola tensor and $\mat{P}^\star$ Piola tensor of the material deformation. 

\newpage
\section{Boundary Conditions}
This sections addresses the setup of the different boundary conditions proposed for the linear momentum $\vect{p}_{\cchi}$, the projected traction $\hat{\vect{t}} = \hat{\mat{P}}\N$, and the material velocity $\vect{w}$.
%
The boundary conditions for the two physical quantities $\vect{p}_{\cchi}, \hat{\vect{t}}$ are defined by the problem's requirements and can be:
\begin{itemize}
	\setlength\itemsep{\lineskip}
	\item free, corresponding to an absence of constraint,
	\item fixed, corresponding to the enforcement of a value picked by the user,
	\item traction, corresponding to a pre-defined applied traction,
	\item roller, corresponding to a symmetric boundary condition
	\item skew-symmetric, corresponding to the use of the opposite operator of the latter.
\end{itemize}
The boundary conditions for $\vect{w}$ are defined by the user, according to the degree of freedom of motion set by the user and can either be:
\begin{itemize}
	\setlength\itemsep{\lineskip}
	\item fixed to 0, corresponding to an absence of motion on the boundary and therefore an exclusively inside motion of the nodes,
	\item roller, corresponding to a possibility for the nodes to slide along their boundary.
\end{itemize}

On the other hand, boundary conditions can be applied in two different ways: Weak boundary conditions refer to enforcement on the faces for the fluxes, and are used in the computation of the boundary terms. Strong boundary conditions are directly applied on the nodes and are called after the update of the variables.

The evaluation of the fluxes is based on a weighted average stencil \cite{Luo1995,Ibrahim2018} defined for a node "a" as 
\begin{equation}
	\vect{F}^\gamma_a = \frac{6\vect{F}_a + \vect{F}_b + \vect{F}_c}{8},
\end{equation}
with b,c the two other nodes that together with node a define boundary face $\gamma$.

\subsection{Free}
\begin{subequations}
\begin{align}
	\hat{\vect{t}}_i^{\gamma} &= \vect{0}
\end{align}
\end{subequations}
\subsection{Fixed}
\begin{subequations}
	\begin{align}
		\hat{\vect{t}}_i^{\gamma} &= \mat{P}_i \N_\gamma
	\end{align}
\end{subequations}
clamp, time varying, 0
\subsection{Traction}
\begin{subequations}
	\begin{align}
		\hat{\vect{t}}_i^{\gamma} &= (\N\otimes\N) \mat{P}_i \N_\gamma
	\end{align}
\end{subequations}
\subsection{Roller}
\subsection{Skew-symmetric}



\newpage
\section{Models}
ALE and solid models are virtually the same 



\newpage 
\section{Time integration}
\begin{subequations}
2 stages Runge-Kutta
\begin{align}
	\vect{U}_a^{\star} &= \vect{U}_a^n + \Delta t \; \dot{\vect{U}}_a^n (\vect{U}_a^n), \\
	\vect{U}_a^{\star\star} &= \vect{U}_a^{\star} + \Delta t \; \dot{\vect{U}}_a^{\star} (\vect{U}_a^{\star}), \\
	\vect{U}_a^{n+1} &= \frac{1}{2} \left( \vect{U}_a^n + \vect{U}_a^{\star\star} \right)
\end{align}
and can also be rewritten:
\begin{align}
	\vect{U}_a^{\star} &= \vect{U}_a^n + \Delta t \; \dot{\vect{U}}_a^n (\vect{U}_a^n), \\
%
	\vect{U}_a^{n+1} &= \frac{1}{2} \vect{U}_a^n + \frac{1}{2} \left( \vect{U}_a^{\star} + \Delta t \; \dot{\vect{U}}_a^{\star} (\vect{U}_a^{\star}) \right) 
\end{align}
3 stages Runge-Kutta
\begin{align}
	\vect{U}_a^{\star} &= \vect{U}_a^n + \Delta t \; \dot{\vect{U}}_a^n (\vect{U}_a^n), \\
%
	\vect{U}_a^{\star\star} &= \frac{3}{4} \vect{U}_a^n + \frac{1}{4}\left( \vect{U}_a^{\star} + \Delta t \; \dot{\vect{U}}_a^{\star}(\vect{U}_a^{\star}) \right), \\ 
%
    \vect{U}_a^{n+1} &= \frac{1}{3} \vect{U}_a^n + \frac{2}{3}\left( \vect{U}_a^{\star\star} + \Delta t \; \dot{\vect{U}}_a^{\star\star}(\vect{U}_a^{\star\star}) \right).
\end{align}
\end{subequations}

\newpage
\section{Plastic Variables}
In Von Mises Model, plasticity variables would be solved using a split approach. Let use define
\begin{equation*}
	\mat{D} = \mat{C}^{p,-1}
\end{equation*}
\begin{subequations}
Remap Stage: (solving $\mat{D}_{n+1}^{\star}$ and $\alpha_{n+1}^{\star}$)
\begin{equation}
	\J_{\psi,n} \left( \frac{\mat{D}_{n+1}^{\star} - \mat{D}_n}{\Delta t} \right) - \left( \GRAD_{\cchi}\mat{D} \right)_n \mat{H}_{\psi,n}^T \vect{w}_{n} = \mat{0},
\end{equation}
and
\begin{equation}
	\J_{\psi,n} \left( \frac{\alpha_{n+1}^{\star} - \alpha_n}{\Delta t} \right) - \left( \GRAD_{\cchi}\alpha \right)_n \cdot \mat{H}_{\psi,n}^T \vect{w}_{n} = 0,
\end{equation}
which can be rewritten as
\begin{equation*}
	\J_{\psi,n} \left( \frac{\alpha_{n+1}^{\star} - \alpha_n}{\Delta t} \right) - 
	\mat{H}_{\psi,n} \left[ \vect{w}_{n} \otimes (\GRAD_{\cchi}\alpha)_n \right] = 0
\end{equation*}
Backward Euler: (solving $\mat{D}_{n+1}$ and $\alpha_{n+1}$)
\begin{equation}
	\frac{\mat{D}_{n+1} - \mat{D}_{n+1}^{\star}}{\Delta t} = -2\dot{\gamma}_{n+1} \left( \mat{F}^{-1} \frac{\partial\phi}{\partial \tau} \mat{b}^e \mat{F}^{-T} \right)_{n+1}
\end{equation}
and
\begin{equation}
	\frac{\alpha_{n+1} - \alpha_{n+1}^{\star}}{\Delta t} = \dot{\gamma}_{n+1}
\end{equation}
\end{subequations}

\newpage 
\subsubsection{Manufactured 2D material mapping with roller BCs}

\begin{equation}
	\vect{\psi}_\beta(\cchi,t) = 
	\begin{pmatrix}
		\chi_1 + \beta sin^2\left(\frac{\pi t}{T}\right) sin\left(\frac{2\pi\chi_1}{\chi_{1,R}}\right) \left[cos\left(\frac{2\pi\chi_2}{\chi_{2,R}}\right)+sin\left(\frac{2\pi\chi_2}{\chi_{2,R}}\right)\right] \\

		\chi_2 + 5*\beta sin^2\left(\frac{2*\pi t}{T}\right) sin\left(\frac{2\pi\chi_2}{\chi_{2,R}}\right) \left[cos\left(\frac{2\pi\chi_1}{\chi_{1,R}}\right)+sin\left(\frac{2\pi\chi_1}{\chi_{1,R}}\right)\right] \\
		
		\chi_3
	\end{pmatrix},
\end{equation}
with $0<\beta<0.02$ the intensity parameter, $T$ the period of the sinusoidal motion (typically indexed on the simulation physical time), $\chi_{1,R}$ and $\chi_{2,R}$ the dimensions of the box.
%
\newpage
\section{Test cases}
\subsection{Plane Strain Tensile}
example 5.1 of \cite{Armero2003}, section 4.3.2 of \cite{Simo1992}, example 3.6 of \cite{DeSouzaNeto1996}, \cite{Simo1985}, \cite{Tvergaard1981}

\bigbreak

True Load vs Displacement
\begin{equation}
	\mat{\sigma}_L = \mat{P}\mat{H}_{\psi}\mat{H}^{-T}
\end{equation}



